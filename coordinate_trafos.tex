\documentclass{scrartcl}

\usepackage{fontspec}

\usepackage{polyglossia}
\setmainlanguage{english}


\usepackage{mathtools}

\usepackage{siunitx}


\usepackage[
  nabla=upright,
  partial=upright,
  bold-style=ISO,
  math-style=ISO,
]{unicode-math}
\setmathfont{Latin Modern Math}

\usepackage{tikz}
\usetikzlibrary{angles}
\usetikzlibrary{quotes}

\usepackage[urldate=edtf, date=edtf]{biblatex}
\addbibresource{references.bib}

\usepackage[section, below, above]{placeins}

\usepackage[colorlinks]{hyperref}
\usepackage{bookmark}
\usepackage[shortcuts]{extdash}

\newcommand\azimuth{\ensuremath{\varphi}}
\newcommand\zenith{\ensuremath{\theta}}

\DeclarePairedDelimiter\norm\lVert\rVert


\begin{document}
  \section{Coordinate System Definition}

  \subsection{Horizontal Coordinate Frame}
  A coordinate in the sky for a given location on earth at a given 
  time is represented using the spherical coordinates \emph{Zenith Distance} \zenith{} and \emph{Azimuth} \azimuth{} or cartesian coordinates $\symbf{r} = (x, y, z)^\top$.
  As only the two angles are needed, we normalize the cartesian coordinates, so that $\norm{\symbf{r}} = 1$.

  The coordinate conventions~\cite{astropy-coords} of \texttt{astropy}~\cite{astropy} are used as shown in \autoref{fig:coords2d}:
  The Azimuth \azimuth{} is $0$ in the North and \ang{90} in the East.
  The $x$\-/axis points North, the $y$\-/axis points East and the $z$\-/axis points upwards.

  \begin{figure}[htpb]
  \centering
  \begin{tikzpicture}

  \draw[thin, lightgray] (0, 0) circle [radius=2cm];
  \foreach \ang in {45, 135, 225, 315} {
    \draw[thin, lightgray, dashed, rounded corners] (0, 0) -- (\ang: 2.8cm)
    node[fill=white] {$\SI{\ang}{\degree}$};
  }

  \coordinate (A) at (0: 2);
  \coordinate (B) at (0, 0);
  \coordinate (C) at (30: 2);

  \draw[->, thick, blue!80!black] (B) -- (C)
  pic [draw=blue!80!black, fill=blue!20, angle radius=1cm, "\azimuth{}"] {angle = A--B--C};

  \draw[->, thick] (-2.2, 0) node[left] {South} -- (2.3, 0) node [right] {$x$  North}; 
  \draw[->, thick] (0, -2.2) node[below] {West} -- (0, 2.3) node [above, align=center, text width=1cm] {East\\$y$}; 

\end{tikzpicture}


  \caption{Definition of the coordinates in the $x$\-/$y$\-/plane according to astropy.}

  \label{fig:coords2d}
  \end{figure}
  

  \printbibliography
\end{document}
